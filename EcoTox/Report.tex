\documentclass[10pt, letter]{report}
\usepackage[utf8]{inputenc}
%\usepackage[frenchb]{babel}
\usepackage[T1]{fontenc}
%\usepackage{a4wide}
\usepackage{indentfirst}
\usepackage{amsfonts}
\usepackage{graphicx,graphics}
\usepackage{amssymb}
\usepackage{rotate}
%\usepackage{pgf,pgfarrows,pgfnodes,pgfautomata,pgfheaps,pgfshade}
\usepackage{multicol}
\usepackage{multirow}
\usepackage{epsf}
\usepackage{epsfig}
\usepackage{amsmath}
\usepackage{psfrag}
\usepackage{color}
%\usepackage{draftcopy}
\usepackage{NOMBRE}
%\usepackage{marvosym}
\usepackage{dcolumn}
\usepackage{subfigure}
\usepackage{color}
\usepackage{fancybox}
\usepackage{fancyhdr}
\usepackage{hyperref}
%\usepackage{xspace}
%\usepackage[breaklinks=true]{hyperref}
%\usepackage{transfig}
\usepackage{lastpage}
\usepackage{placeins}

\pagestyle{headings}
%\usepackage{palatino}
\topmargin=-1cm
\rightmargin=0cm
\headheight=1cm
\headsep=1cm
\footskip=1cm
\textheight=21.5cm
\textwidth=16.4cm
\oddsidemargin=-0.5cm
\evensidemargin=-0.5cm
%\usepackage{braket}
%\newtheorem{theo}{Théorème}
%\bigskip
\def\Tiny{\fontsize{3pt}{3pt}\selectfont}



\newcolumntype{M}[1]{>{\raggedright}m{#1}}

\newcommand{\ket}[1]{\ensuremath{\left| #1 \right\rangle}}
\newcommand{\bra}[1]{\ensuremath{\left\langle #1 \right|}}
\newcommand{\vect}[1]{\ensuremath{\overrightarrow{#1}}}
\newcommand{\encadre}[1]{\begin{equation}
\begin{tabular}{|c|}
\hline
\\
$\displaystyle \; #1 \;$\\
\\ \hline
\end{tabular}
\end{equation}}


\newcommand{\encadredeux}[1]{\begin{equation}
\begin{tabular}{|c|}
\hline
\\
$\displaystyle \; #1 \;$\\
\\ \hline
\end{tabular}\nonumber
\end{equation}}


\newcommand{\encadrebut}[1]{\begin{equation}
\begin{tabular}{|c|}
\hline
\\
$\displaystyle \; \mbox{\underline{But} :  #1} \;$\\
\\ \hline
\end{tabular}\nonumber
\end{equation}}


\newcommand{\moy}[1]{\ensuremath{\left\langle #1 \right\rangle}}
\newcommand{\ave}[1]{\ensuremath{\bar{#1}}}



\title{Modélisation Silice}
\date{Juillet 2012}
\author{Jean-Yves DELANNOY}

\newcommand{\f}[2]{{\ensuremath{\mathchoice%
        {\dfrac{#1}{#2}}
        {\dfrac{#1}{#2}}
        {\frac{#1}{#2}}
        {\frac{#1}{#2}}
        }}}

%\bibliographystyle{plain}
%\bibliography{bibli.bib}


\renewcommand{\=}{\, =\, }
\newcommand{\+}{\, +\, }
\renewcommand{\-}{\, -\, }
\newcommand{\vv}{\vspace{-0.5cm}}

\newcommand{\trip}[3]{\ensuremath{\left[T_{#1},\left[T_{#2},T_{#3}\right]\right]}}

\newcommand{\eqn}[1]{\begin{equation} #1 \end{equation}}


\newcommand{\property}[2]{{\underline{Propriété #1}} : 
{\it \bf #2}
}

\newcommand{\ie}{{\it i.e. }}
\newcommand{\cf}{{\it cf.~}}

\newcommand{\bc}[1]{\begin{tabular}{|c|}
\hline
#1\\
\hline\end{tabular}_{p\times p}}

\newcommand{\entier}[1]{\left[\hspace{-1ex}\left[\hspace{0.5ex}#1\hspace{0.5ex} \right]\hspace{-1ex}\right]}





%%%%%%%%%%%%%%%%%%%%%%%%%%%%%%%%%%%%%%%%%%%%%%%%%%%%%%%%%%%%
% Show subsubsection numbers as a letter
\setcounter{secnumdepth}{3}
\makeatletter
\renewcommand\thesubsubsection{\thesubsection .\@alph \c@subsubsection}
\makeatother

%%%%%%%%%%%%%%%%%%%%%%%%%%%%%%%%%%%%%%%%%%%%%%%%%%%%%%%%%%%%
% make graph and figure scaling uniform

\newcommand{\setpath}[1]{
  \graphicspath{{figures/#1/}{graphes/#1/}}
}

\newcommand{\graphscale}{0.27}
\newcommand{\figscale}{0.6}

\newcommand{\includefigure}{\includegraphics}
\newcommand{\includegraph}{\includegraphics}

\newcommand{\includescaledgraph}[1]{%
  \includegraphics[scale=\graphscale]{#1}%
}

\newcommand{\includescaledfigure}[1]{%
  \includegraphics[scale=\figscale]{#1}%
}


\newlength{\espace}
\setlength{\espace}{0.8cm}
\setlength{\footskip}{1cm}
\definecolor{gris}{gray}{0.5}
%\graphicspath{{../}{C:/Users/jdelanno/Documents/Presentation/IMAGES/}{C:/Users/jdelanno/Documents/COMPNANOCOMP/Deliverables/}}


%\graphicspath
%\usepackage{fancyhdr}
\setlength{\headheight}{25pt}

 
\pagestyle{fancy}
%\renewcommand{\chaptermark}[1]{\markboth{#1}{}}
\renewcommand{\sectionmark}[1]{\markright{#1}{}}
\renewcommand{\headrulewidth}{0pt} % remove lines as well
\renewcommand{\footrulewidth}{0pt}

\fancyhf{}
\fancyfoot[LE,RO]{\thepage}
%\fancyfoot[RE]{\textit{\rightmark}}
%\fancyfoot[LO]{\textit{\rightmark}}
%\fancyfoot[RE,LO]{\textit{\rightmark}\\
%\vspace*{1.5cm}
%\textcolor{gris}{\Tiny Centre de Recherches et Technologies de Lyon : 85, rue des Frères Perret. BP 62. F-69192 Saint-Fons. Tél. : +33 4 72 89 67 89. Fax : +33 4 72 89 68 63\\
%Siret : 622 037 083 00285\\
%Dénomination sociale : Rhodia Opérations. 40, rue de la Haie Coq. 93306 Aubervilliers Cedex. France. Tél. : + 33 1 53 56 50 00. Fax : + 33 1 53 56 55 55\\
%Société par Actions Simplifiée au capital de 695 897 850 euros. RCS Bobigny 622 037 083. TVA intracommunautaire 41 622 038 083\\
%{\bf www.rhodia.com}}
%}

\fancyhead[LE,RO]{\hspace*{2cm}\includegraphics[width=0.08\textwidth]{Logo_Solvay}}

\usepackage[final]{pdfpages}


\newcommand{\degre}{$^{\circ}\mathrm{C}\,$}


\begin{document}


\title{\textbf{COMPNANOCOMP Final Report.}}
%treatment


\author{P. J. Kowalczyk}
\date{\today}

\vspace*{1cm}
\hspace*{-1cm}\begin{tabular}{p{0.49\textwidth}p{0.08\textwidth}p{0.42\textwidth}}
{\bf Date:} \today & \multicolumn{2}{r}{{\huge \bf{Technical Report }}}\\
\\
\hline
\\
{\bf De} : Paul Kowalczyk  & \bf{\`A:} & Jean-Yves Delannoy \\
&& Alessio Tamburro \\

 \\
{\bf Copie} :   \\
{\bf Ref} :&  {\bf Pages : }  & \pageref{LastPage} \\
\\
\hline
\\
\multicolumn{3}{c}{\LARGE In Silico Prediction of Physicochemical Properties} \\
\\
\multicolumn{3}{c}{\LARGE Final Report.} \\
\\
\hline
\end{tabular}
%
\vspace*{2cm}

Quantitative structure-property relationship (QSPR) models were developed for the
prediction of six physicochemical properties of environmental chemicals: octanol–water
partition coefficient (log P), water solubility (log S), boiling point (BP), melting point (MP),
vapor pressure (VP) and bioconcentration factor (BCF). Models were developed using
simple binary molecular fingerprints and four approaches with differing complexity: multiple
linear regression, random forest regression, partial least squares regression, and support
vector regression (SVR). To obtain reliable and robust regression models with high
prediction performance, genetic algorithms (GA) were employed to select the most
information-rich subset of fingerprint bits. Predictions from the various models were tested
against a validation set, and all four approaches exhibited satisfactory predictive results, with
SVR outperforming the others. BP was the best-predicted property, with a correlation
coefficient (R\(^{2}\)) of 0.95 between the estimated values and experimental data on the
validation set while MP was the most poorly predicted property with an (R\(^{2}\)) of 0.84. The
statistics for other properties were intermediate between MP and BP with (R\(^{2}\)) equal to 0.94,
0.93, 0.92 and 0.86 for log S, log P, VP and BCF, respectively. The prediction results for all
properties were superior to those from Estimation Program Interface (EPI) Suite (R\(^{2}\) values
ranged from 0.63 to 0.94), a widely used tool for property prediction. This study
demonstrates that (1) molecular fingerprints are useful descriptors, (2) GA is an efficient
feature selection tool from which selected descriptors can effectively model these properties,
and (3) simple methods give comparable results to more complicated methods. 

It's hoped that in describing how one might build predictive models for chemical datasets, colleagues will be encouraged to discuss how these same workflows may be applied to Solvay project data.


\vspace*{1cm}
\begin{tabular*}{5.03\textwidth}{lr}
%\includegraphics[height=1.5cm,width=4cm]{signature_JY} & \includegraphics[height=1.5cm,width=4cm]{signature_F_clement} \\
Paul KOWALCZYK &\hspace*{9cm} Jean-Yves DELANNOY\\
\end{tabular*}

\tableofcontents


\chapter{Introduction}

\begin{itemize}
\item Current tools for testing the biological activity and toxicity of chemicals are
time-consuming and costly. Thus, only a fraction of these chemicals have been
fully characterized for their potential hazard and risks to both human health and
the environment.
\item In vitro and in silico approaches are being developed as more efficient tools for
chemical hazard characterization and prioritization. One of these approaches is in
silico estimation of physicochemical properties.
\item This study presents novel methods using simple binary molecular fingerprints for
the estimation of six physicochemical properties of environmental chemicals:
\begin{itemize}
\item Octanol–water partition coefficient (log P)
\item Water solubility (log S)
\item Boiling point (BP)
\item Melting point (MP)
\item Vapor pressure (VP)
\item Bioconcentration factor (BCF)
\end{itemize}
\item The goal of this project is to produce models that can be easily used by Solvay colleagues and that adhere to
internationally accepted validation principles defined by the Organisation for
Economic Co-operation and Development (OECD 2004).
\end{itemize}

\chapter{Characteristics of the Chemical Sets}

\begin{itemize}
\item Experimentally measured physicochemical properties of a structurally diverse set
of organic environmental chemicals were obtained from EPI Suite (EPA 2012 and
EPI Suite Data). These chemicals represent a wide range of use classes,
including industrial compounds, pharmaceuticals, pesticides, and food additives.
\item Figure 1 shows that values for the physicochemical properties of the chemical
set are normally or nearly normally distributed.
\begin{itemize}
\item Log P (Figure 1a) ranges from -4.27 to 8.54 log units with a median of 2.19.
\item Log S (Figure 1b) ranges from -9.70 to 1.58 log units (mol/L) with a median
of -2.38.
\item BP (Figure 1c) ranges from -88.60 to 548.00 C with a median of 189.20 C.
\item MP (Figure 1d) ranges from -199.00 to 385.00 C with a median of 85.00 C.
\item VP (Figure 1e) ranges from -13.68 to 5.89 log units (mmHg) with a median of
-2.11.
\item BCF (Figure 1f) ranges from -0.35 to 5.97 log units with a median of 1.73.
\end{itemize}
\end{itemize}

\chapter{Definition of Training and Test Sets}

The chemicals were randomly partitioned into training sets (80\% of the
chemicals) to build the models and test sets (20\% of the chemicals) to validate
the predictive power of each model. Table 1 lists the summary statistics for physicochemical properties of the training
and test sets.


\chapter{Development of QSPR Models}

\begin{itemize}
\item Molecular fingerprints, a series of binary bits that represent the presence (1) or
absence (0) of particular substructures in a molecule, were used as independent
variables.
\item Genetic algorithm (GA; Wegner et al. 2003) was employed to select the most
information-rich subset of variables for obtaining reliable and robust regression
models.
\item Quantitative structure–property relationship (QSPR) models were developed
using four approaches with differing complexity in ascending order: multiple linear
regression (MLR), partial least squares regression (PLSR), random forest
regression (RFR), and support vector regression (SVR).
\item Mathematical processing for data standardization, multivariate regression
analysis, and statistical model building were performed using the statistical
software package R (version 3.0.2)(R Development Core Team 2008). GA, MLR,
RFR, PLSR and SVR were implemented by the packages subselect, stats,
randomForest, pls and e1071, respectively.
\item The performance of each QSPR model is evaluated by establishing a correlation
between the experimental and calculated values with a set of parameters:
\begin{itemize}
\item R\(^{2}\) and RMSE are the coefficient of determination and root mean squared
error for training or test set with n chemicals.
\item Q\(^{2}\) and RMSEcv are the coefficient of determination and root mean squared
error for 10-fold cross validation (CV) with v chemicals not included in the CV
\end{itemize}
\end{itemize}

\textbf{Molecular Fingerprints.} The chemicals were represented
by fingerprints derived from their molecular structures.
Fingerprints were calculated using a wide variety of publicly
available SMARTS systems implemented in PaDEL:51,52 Estate
(79 bits), Extended (1024 bits), Substructure (307 bits),
Klekota Roth (4860 bits), PubChem (881 bits), Atom Pairs 2D
(780 bits), and MACCS (166 bits). A total of 8097 binary bits
were generated, with 1 and 0 denoting the presence or absence,
respectively, of a specific structural fragment. Fingerprint bits
with zero variance (i.e., uniform observations across the set)
were removed. To obtain reliable models, sufficient occurrences
of the fingerprint bits throughout the entire data sets are
necessary and thus, bits with low occurrences (<2%) were
eliminated. Following removal of highly correlated and
infrequently occurring bits, the resulting numbers of bits
retained and employed to build the regression models were:
1681 for logP; 1061 for logS; 450 for logBCF; 1050 for BP;
1424 for MP; and 1145 for logVP. A genetic algorithm
(GA)53,54 was used to reduce the feature space by assigning an
initial population of chromosomes to two times the number of
variables (fingerprint bits). The crossover probability on each
chromosome in a population and mutation rate on each gene in
a chromosome were set to 50\% and 1\%, respectively. There
were no improvements in the fitness score after 1000
generations.

\textbf{Multiple Linear Regression.} Multiple linear regression
(MLR) is widely used in the modeling of property data.40,55 We
used MLR to produce a linear model to describe the
relationship between a physicochemical property and the
molecular fingerprint bits:
\begin{equation}
property = \sum_{j = 1}^{m}c_{j}f_{j}
\end{equation}
In eq 4.1, property is one of the six physicochemical properties
(logP, logS, logBCF, BP, MP or logVP); cj is the contribution
coefficient, which is determined by regression analysis; and f j is
the binary bit of the jth fingerprint, with its presence or absence
represented by the numeric values 1 or 0 respectively. Any
fragment that occurred in a molecule was counted only once for
that molecule, no matter how many times it occurred in the
molecule.

\textbf{Partial Least Squares Regression.} Partial least-squares
regression (PLSR) is a widely used multivariate analytical
technique in QSPR studies.56,57 The advantage of PLSR over
MLR lies in its ability to build a regression model based on
highly correlated descriptors, extract the relevant information,
and reduce data dimensions. We employed PLSR to generate
linear statistical models based on the fingerprint bits and the
physicochemical property being predicted. A set of orthogonal
latent variables or principal components (PCs) were first
generated through a linear combination of the original
molecular fingerprint bits, which served as new variables for
regression with the response variables (i.e., the physicochemical
properties) to build QSPR models. The optimal number of PCs
was determined by 10-fold cross-validation (CV).

\textbf{Random Forest Regression.} Random forest (RF) is a
nonlinear consensus method based upon an ensemble of
decision trees which are grown from separate bootstrap samples
of the training data.58 Bootstrap sampling is conducted via
random selection with replacement from the training chemicals
during tree growth. The chemicals that are not selected in the
construction of the forest are called out-of-bag (OOB) samples,
which are used to evaluate the prediction accuracy as trees are
added to the forest. Each tree gives a prediction for its OOB
chemicals, and the average of these results over all trees
provides an overall unbiased external validation. There are three
possible model parameters for RF regression: ntree - the
number of trees in the forest; mtry - the number of variables
randomly sampled at each tree node; and nodesize - the
minimum node size below which nodes are not further
subdivided. In the present study, the RF model was trained
based upon a parameter combination of ntree = 500, nodesize
= 5, and mtry = 1/3 the number of fingerprint bits.

\textbf{Model Validation.} The performance of each QSPR model
was evaluated by examining the correlation between the
experimental and predicted values using the following
parameters:61,62 R2 (coefficient of determination) and RMSE
(root mean squared error) for training or test sets with n
chemicals; Q2 (coefficient of determination) and RMSEcv for
10-fold CV with v chemicals not included in the CV model
building set. The 10-fold CV procedure was completed using
only the training set.

\begin{equation}
R^{2} = 1 - \frac{\sum_{i = 1}^n(p_{i}-\hat{p}_{i})^{2}}{\sum_{i = 1}^n(p_{i}-\bar{p})^{2}}
\end{equation}

\begin{equation}
Q^{2} = 1 - \frac{\sum_{i = 1}^\nu(p_{i}-\hat{p}_{i})^{2}}{\sum_{i = 1}^\nu(p_{i}-\bar{p})^{2}}
\end{equation}

\begin{equation}
RMSE = \sqrt{\frac{1}{n}\sum_{i = 1}^n(p_{i}-\hat{p}_{i})^{2}}
\end{equation}

\begin{equation}
RMSE_{cv} = \sqrt{\frac{1}{\nu}\sum_{i = 1}^\nu(p_{i}-\hat{p}_{i})^{2}}
\end{equation}

In eqs 4.2 -- 4.5, \(p_{i}\) and \(\hat{p}_{i}\)
are the measured and predicted property
values for chemical \textit{i}, respectively, and \( \hat{p} \) is the mean of all
chemicals in the data set. In addition, standard error of
prediction (SEP) was employed as a criterion to select the
optimal principal components in the PLSR analysis.

\begin{equation}
SEP = \sqrt{\frac{1}{n-1}\sum_{i = 1}^n(p_{i}-\hat{p} - bias)}
\end{equation}

\begin{equation}
bias = \frac{1}{n}\sum_{i = 1}^{n}(p_{i}-\hat{p})
\end{equation}

\textbf{Applicability Domain.} Three distance-based measures, i.e.,
leverage, distance from centroid and k-nearest neighbors
(kNN), were applied to assess the applicability domain (AD)
of each regression model. The distance of a test chemical from
a defined point in the descriptor space of the training set was
calculated and compared to a predefined threshold. The
test chemical is considered to be within AD if its distance is less
than or equal to the threshold. Leverage is defined as the
diagonal element of the covariance matrix for a given data set,
and the leverage of a test chemical is proportional to Hotellings
T2 statistic and its Mahalanobis distance. The threshold was set
to three times the average of the leverage (3\(\frac{m}{n}\), with \textit{m} being
the number of variables and n the number of training
chemicals). For the measure of distance from centroid, the
distance of a test chemical from the training set centroid is
compared with a threshold, which is determined as follows: (1)
calculate the distances of training chemicals from their centroid;
(2) sort the vector of distances in ascending order; (3) set the
distance value corresponding to 95th percentile as the
threshold. The kNN measure defines the model’s AD based
on the similarity between a test chemical and the training
chemicals. The average distance of the test chemical from its
five nearest neighbors in the training set is compared with a
threshold, which is the 95th percentile of average distance of
training chemicals from their five nearest neighbors.

\textbf{Statistical Analysis.} Mathematical processing for data
standardization, multivariate regression analysis, and statistical
model building were performed using the R statistical
computing environment for Windows (version 3.2.1).66
Genetic algorithm, multiple linear regression, partial leastsquares
regression, random forest regression, support vector
regression and distance of k-nearest neighbors were implemented
by the R packages subselect, stats, pls, randomForest,
e1071, and FNN, respectively. The R code for feature selection
and regression analysis is provided in the Supporting
Information.

Pearson correlation coefficients (\textit{r})
\begin{equation}
r = \frac{n\sum{p_{k}p_{l}} - \sum{p_{k}}\sum{p_{l}}}{\sqrt{n\sum{p_{k}^{2}} - (\sum{p_{k}})^{2}\sqrt{n\sum{p_{l}^{2}} -(\sum{p_{l}})^{2}}}}
\end{equation}


\chapter{Correlation Between Estimated and Measured Values}



The property of a chemical calculated from a set of molecular fingerprints can be
described by a general equation:
\begin{equation}
property = \sum_{j = 1}^{m}c_{j}f_{j}
\end{equation}
In equation (5.1):
\begin{itemize}
\item textit{property} is the value of the physicochemical property
\item \(c_{j}\) is the contribution coefficient, which is determined by regression analysis
\item \(f_{j}\) is the binary bit of the jth fingerprint, with presence or absence denoted by
the numeric value 1 or 0
\end{itemize}
The quality of the model depends heavily on the number of selected fingerprint
bits, and the predictive performance of the model is enhanced remarkably when
an appropriate number of fingerprint bits were selected from GA (Figure 2).
Results show that the prediction for the training set is improved continuously with
increasing feature number. In contrast, the test set followed a different pattern,
\textit{i.e.}, the RMSE value initially decreased, attained a minimum at a medium number
of bits, and then gradually increased afterwards.
\begin{itemize}
\item For log P, the modeling statistics are not sensitive to the bit number, and the
model performance does not vary considerably with different subsets of
fingerprint bits for the test set (Figure 2a).
\item For log S, the lowest prediction errors occurred on the models with moderate
complexity around 250 and 300 bits
\end{itemize}

The validation results show a significant correlation between the estimated and
measured values in the test set.
\begin{itemize}
\item For log P, R\(^{2}\) of 0.925 corresponded to a minimum RMSE of 0.516 log units
for test set when using 600 fingerprint bits selected by GA, compared to R2 of
0.980 for training set (Figure 3a).
\item For log S, R\(^{2}\) of 0.935 corresponded to a minimum RMSE of 0.559 log units
for test set when using 250 fingerprint bits selected by GA, compared to R2 of
0.955 for training set (Figure 3b).
\end{itemize}

\chapter{Relationship Between Number of Principal Components and Standard Error of Prediction}

\begin{itemize}
\item The number of significant principal components (PCs) for the PLS algorithm was
determined using 10-fold cross-validation (CV) procedure on the training set
(Zang et al. 2011). The relation of the standard error of prediction (SEP) versus
the number of PCs is displayed in Figure 4.
\item The gray lines were produced by repeating this procedure 100 times. The
black line represents the lowest SEP value from a single 10-fold CV. The
dashed vertical lines represent the optimal number of PCs and the dashed
horizontal lines indicate the SEP value for the test set when the optimal PCs
are applied.
\item For the all-descriptor model, initially SEP decreases with PCs, and then starts
to rebound after a certain point when the model begins to simulate the noise
as the complexity of the model increases (Figure 4a). For the 600-bit model,
the SEP decreases monotonically and gradually approaches a stable value,
and the model with 42 PCs gave a minimum RMSE (Figure 4b).
\end{itemize}

\chapter{Applicability Domain}

\begin{itemize}
\item An applicability domain (AD) is a chemical, structural, or physicochemical space of
the training set.
\item The AD of the models was assessed using a leverage-based approach that
compares a predefined threshold to the distance of query compounds from a
defined point within the descriptor space. The approach is based on the
covariance matrix derived from center-scaled variables. The threshold is three
times the average of the leverage that corresponds to m/n, the ratio of m, the
number of model variables, to n, the number of training compounds.
\item Figure 5 displays the relationship between leverage and standardized residuals
(William plot [Sahigara et al. 2012]).
\begin{itemize}
\item For log P, 39 out of 2998 (1.30\%) test chemicals are located outside the AD
(Figure 5a).
\item For log S, 18 out of 457 (3.94\%) test chemicals are located outside the AD
(Figure 5b).
\end{itemize}
\end{itemize}

\chapter{Comparison of the Models}

SVR substantially outperformed the other three approaches in predicting log P, log
BCF, BP and MP with a low error rate (Table 3). However, performance of SVR
was similar to the other three approaches for predicting log S and log VP.

\chapter{Conclusions}

This study demonstrates that:
\begin{itemize}
\item Molecular fingerprints are useful descriptors for modeling the six properties.
\item GA is an efficient feature selection tool from which selected descriptors can
effectively model these properties.
\item Simple methods such as MLR give similar results to more complicated
methods under optimal conditions for modeling log S and log VP.
\item There are multiple ways for deriving regression models with similar statistics.
\item When compared to other procedures currently in use, these methods present better
accuracy for a wider range of chemicals of interest, are highly stable and reliable,
and are in line with the validation principles put forth by the OECD. They thus
have broad applicability for property estimation of many classes of compounds.
\end{itemize}

\newpage
\begin{center}
{\bf FICHE RESUME/BIBLIOGRAPHICAL FORM DOCUMENTUM}
\end{center}
\begin{flushleft}
\begin{tabular}{|l|p{0.18\textwidth}|p{0.63\textwidth}|}
\hline
&&\\
{\bf SERVICE} & Laboratory Group & SM@RT\\
&&\\ \hline &&\\
{\bf Type de document} & Document type & Rapport de Projet \\
&&\\ \hline &&\\
\bf Date & Application Date & Decembre 2014\\
&&\\ \hline &&\\
{\bf TITRE en Anglais} & English Title & COMPNANOCOMP FP7 Project : Final Report \\
&&\\ \hline &&\\
{\bf TITRE en Français} & French Title & Projet FP7 COMPNANOCOMP : Rapport Final \\
&&\\ \hline &&\\
\bf Entreprise & Enterprises/Sponsors & R$\&$I/AIO/Advanced Materials Platform  \\
&&\\ \hline &&\\
\bf Auteur(s) & Authors & Jean Yves DELANNOY \\
&&\\ \hline &&\\
\bf PROJET & Projects & COMPNANOCOMP\\
&&\\ \hline &&\\
\bf Collaborateurs & Collaborators & Cédric Feral-Martin\\
&& Aur\'elie Papon\\
&& Magali Fontana\\
&& Olivier Sanseau \\
&&\\ \hline &&\\
\bf N$^0$ d'affaire & Business code & \\
&&\\ \hline &&\\

\bf RESUME  Anglais & & The objective of this document is to offer a "digest" of the main results obtained  in this project that aims at the development of multiscale simulation methodology and software for predicting the morphology (spatial distribution and state of aggregation of nanoparticles), thermal (glass temperature), mechanical (viscoelastic storage and loss moduli, plasticity, fracture toughness and compression strength), electrical and optical properties of soft and hard polymer matrix nanocomposites from the atomic-level characteristics of their constituent nanoparticles and macromolecules and from the processing conditions used in their preparation.

The document gives a short summary of the results obtained within this project by Solvay and its partners. It also emphasizes the interest for Solvay R$\&$I of  the developments obtained. 


\\
&&\\ 
\hline
\end{tabular}

\begin{tabular}{|l|p{0.18\textwidth}|p{0.63\textwidth}|}
\hline
&&\\
\bf RESUME Français &&  L'objectif de ce rapport est de fournir un résumé des points essentiels développés dans le rapport final du projet européen FP7 COMPNANOCOMP dont le but est de développer une méthodologie de simulation multi-échelle permettant de prédire la morphologie (distribution spatiale et état d'agrégation des nanoparticules), et les propriétés thermiques, mécaniques optiques et électriques de nanocomposites de polymères. Ce travail s'effectue sur la base des caractéristiques atomiques des constituants du matériaux et doit prendre en compte les conditions du procédé ayant permis sa réalisation.

Ce document donne une vision des résultats obtenus par Solvay et ses collaborateurs et met en avant l'intérêt pour le groupe des développements effectués.
\\
&&\\ \hline &&\\
\bf Mots Clés Anglais & English Keywords&  RUBBER, REINFORCEMENT, SILICA, COUPLING, MODELING, MORPHOLOGY, FP7\\
&&\\ \hline &&\\
\bf Mots Clés Français &&  CAOUTCHOUC ; RENFORT ; SILICE ; COUPLAGE ; MODELISATION ; MORPHOLOGIE, FP7\\
&&\\ \hline &&\\
\bf RNCAS & RNCAS & \\
&&\\ \hline &&\\
\bf Destinataires& Addressees & \\
&&\\ \hline &&\\
\bf CONFIDENTIEL & Confidential & YES \\
&&\\ \hline
\end{tabular}



\end{flushleft}




\end{document}





