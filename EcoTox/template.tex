Quantitative structure-property relationship (QSPR) models were developed for the
prediction of six physicochemical properties of environmental chemicals: octanol–water
partition coefficient (log P), water solubility (log S), boiling point (BP), melting point (MP),
vapor pressure (VP) and bioconcentration factor (BCF). Models were developed using
simple binary molecular fingerprints and four approaches with differing complexity: multiple
linear regression, random forest regression, partial least squares regression, and support
vector regression (SVR). To obtain reliable and robust regression models with high
prediction performance, genetic algorithms (GA) were employed to select the most
information-rich subset of fingerprint bits. Predictions from the various models were tested
against a validation set, and all four approaches exhibited satisfactory predictive results, with
SVR outperforming the others. BP was the best-predicted property, with a correlation
coefficient (R2
) of 0.95 between the estimated values and experimental data on the
validation set while MP was the most poorly predicted property with an R2 of 0.84. The
statistics for other properties were intermediate between MP and BP with R2 equal to 0.94,
0.93, 0.92 and 0.86 for log S, log P, VP and BCF, respectively. The prediction results for all
properties were superior to those from Estimation Program Interface (EPI) Suite (R2 values
ranged from 0.63 to 0.94), a widely used tool for property prediction. This study
demonstrates that (1) molecular fingerprints are useful descriptors, (2) GA is an efficient
feature selection tool from which selected descriptors can effectively model these properties,
and (3) simple methods give comparable results to more complicated methods.
